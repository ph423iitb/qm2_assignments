\documentclass{article}
\usepackage{amsmath}
\usepackage{amssymb}
\usepackage[utf8]{inputenc}
\usepackage{geometry}
\geometry{margin=1 in}


\title{Assignment 1}
\author{Mytraya Gattu, Roshni Singh, Ramakant Pal\\ 180050032, 18B030020, 180260028}
\date{\today}

\begin{document}

\maketitle

\section*{Problem 1}
Given that $\langle x^{\prime}|x \rangle = \delta(x^{\prime}-x)$, we can write the operator X in the position basis (its own eigenbasis) as:
$$\langle x^{\prime} | X | x \rangle = x\delta\left(x^{\prime}-x\right)= x^{\prime}\delta\left(x^{\prime}-x\right)$$ We also take the canonical commutation relation as one of the postulates: $$\left[X,P\right] = \imath \hbar$$
Now we can write:
$$\langle x^{\prime} | \left[X,P\right] | x \rangle = \imath \hbar \delta\left(x^{\prime}-x\right)$$
$$\implies \langle x^{\prime} | XP| x \rangle - \langle x^{\prime} | PX| x \rangle = \imath \hbar \delta\left(x^{\prime}-x\right)$$
$$\implies x^{\prime}\langle x^{\prime} | P| x \rangle-x\langle x^{\prime} | P| x \rangle=\imath \hbar \delta\left(x^{\prime}-x\right)$$
Therefore, 
$$\langle x^{\prime} | P| x \rangle = \imath \hbar \frac{\delta\left(x^{\prime}-x\right)}{x^{\prime}-x} = -\imath \hbar \frac{d}{dx}\left(\delta \left(x-x^{\prime}\right) \right)$$
\section*{Problem 2}
In the ket representation, 
$$\partial_{t} |\psi(t) \rangle = -\frac{\imath H(t)}{\hbar} | \psi (t) \rangle$$
The wave function $\psi (x,t) $ is defined as $\langle x | \psi (t) \rangle$.
Multiplying the equation by $\langle x|$ on both sides we get:
$$\partial_{t} \langle x|\psi(t) \rangle = -\imath \langle x | \frac{H(t)}{\hbar} | \psi (t) \rangle$$
$$\implies \partial_{t} \psi(x,t) = -\imath \langle x | \frac{H(t)}{\hbar} | \psi (t) \rangle$$
We now insert the identity $\int dx^{\prime} | x^{\prime} \rangle \langle x^{\prime} | = 1$, to get
$$\partial_{t}\psi(x,t)=-\frac{i}{\hbar}\int dx^{\prime} \langle x | H(t) | x^{\prime} \rangle \langle x^{\prime}|\psi (t) \rangle$$
$$\implies \partial_{t}\psi(x,t)=-\frac{i}{\hbar}\int dx^{\prime} \psi(x^{\prime},t) \langle x | H(t) | x^{\prime} \rangle$$
Now, we assume that $H(t)$ is given by $H(t) = \frac{P^2}{2m} + V$.
$$ \langle x | H(t) | x^{\prime} \rangle =  \langle x | \frac{P^2}{2m} + V | x^{\prime} \rangle = \langle x | \frac{P^2}{2m}  | x^{\prime} \rangle + \langle x | V  | x^{\prime} \rangle$$
We use the position space representation of  the momentum and potential operators to get:
$$ \langle x | H(t) | x^{\prime} \rangle =  [-\hbar^2 \frac{\partial^2}{\partial x^2} + V(x) ] \delta(x-x^{\prime})$$


Using this, therefore
$$\partial_{t}\psi(x,t)=-\frac{i}{\hbar}\int dx^{\prime}  \psi (x^{\prime},t)  [-\hbar^2\frac{\partial^2}{\partial x^2} + V(x) ] \delta(x-x^{\prime})$$

Therefore, we finally get:
$$ -\frac{\hbar}{\imath}\frac{\partial}{\partial t}\psi(x,t) = (-\frac{\hbar^2}{2m}\frac{\partial^2}{\partial x^2} + V(x))\psi(x,t)$$
(Schrodinger's Equation)
\section*{Problem 3}
For a free non-relativistic particle, $$H(t) = \frac{P^2}{2m}$$
Using the operator identity in the momentum space (taking the normalisation of plane wave states to be $\frac{1}{\sqrt{(2 \pi \hbar)^{D}}}$ in $D$ dimensions), we have
$$ \langle x^{\prime} | e^{-\frac{\imath}{\hbar} Ht} | x \rangle = \int dp \langle x |p \rangle  \langle p | e^{-\frac{\imath}{2\hbar m}P^2t} | x^{\prime} \rangle $$
Simplifying,
$$ \langle x^{\prime} | e^{-\frac{\imath}{\hbar}Ht} | x \rangle = \int dp \frac{1}{2\pi\hbar} e^{\frac{i}{\hbar}p\left(x-x^{\prime}\right)-\frac{\imath}{2\hbar m}p^2t}$$
(Since, $\langle x | p \rangle = \sqrt{\frac{1}{2\pi \hbar}}e^{\frac{\imath}{\hbar} p\cdot x}$) \newline
Resolving the integral, (by completing the square in the exponent):
$$ \langle x^{\prime} | e^{-\frac{\imath}{\hbar}Ht} | x \rangle =\sqrt{\frac{m}{2 \imath t \pi \hbar}} e^{\frac{i m (x-x^{\prime})^2}{2 \hbar t}}$$
\subsection*{Working}
The power of e is:
$$
-\frac{i}{\hbar}\left[\frac{p^{2} t}{2 m}-p\left(x-x^{\prime}\right)\right]
$$
Putting \( \alpha^{2}=\frac{t}{2 m} \), we get

$$
\frac{-i}{\hbar}\left[\alpha^{2} p^{2}-2(\alpha p) \frac{\left(x-x^{\prime}\right)}{2 \alpha}+\frac{\left(x-x^{\prime}\right)^{2}}{4 \alpha^{2}}\right]+\frac{i}{\hbar}\left[\frac{(x-x^{\prime})^2}{4 \alpha^{2}}\right]
$$

which can be written as:

$$
-\frac{i}{h} \quad\left(\alpha p-\frac{\left(x-x^{\prime}\right)}{2 \alpha}\right)^{2}+\frac{i}{\hbar} \frac{\left(x-x^{\prime}\right)^{2}}{4 \alpha^{2}}
$$

Therefore we can rewrite the integral as:
\( \left\langle x^{\prime}\left|e^{-\frac{\imath}{\hbar} H t}\right| x\right\rangle=\int d p \frac{1}{2 \pi \hbar}e^{-\frac{i}{h} \quad\left(\alpha p-\frac{\left(x-x^{\prime}\right)}{2 \alpha}\right)^{2}+\frac{i}{\hbar} \frac{\left(x-x^{\prime}\right)^{2}}{4 \alpha^{2}}} \)

Therefore, we can write our integral as,

\( \left\langle x^{\prime}\left|e^{-\frac{\imath}{\hbar} H t}\right| x\right\rangle=\int d p \frac{1}{2 \pi \hbar} e^{-\frac{i}{h} \quad\left(\alpha p-\frac{\left(x-x^{\prime}\right)}{2 \alpha}\right)^{2}} e^{\frac{i}{\hbar} \frac{\left(x-x^{\prime}\right)^{2}}{4 \alpha^{2}}} \)

% \( \left\langle x^{\prime}\left|e^{-\frac{\imath}{\hbar} H t}\right| x\right\rangle=\int d p \frac{1}{2 \pi \hbar} e^{-\frac{i}{h} \quad\left(\alpha p-\frac{\left(x-x^{\prime}\right)}{2 \alpha}\right)^{2}} e^{\frac{i}{\hbar} \frac{\left(x-x^{\prime}\right)^{2}}{4 \alpha^{2}}} \)

\( \left\langle x^{\prime}\left|e^{-\frac{\imath}{\hbar} H t}\right| x\right\rangle= e^{\frac{i}{\hbar} \frac{\left(x-x^{\prime}\right)^{2}}{4 \alpha^{2}}}  \frac{1}{2 \pi \hbar} \int d p e^{-\frac{i}{h} \quad\left(\alpha p-\frac{\left(x-x^{\prime}\right)}{2 \alpha}\right)^{2}}  \)


The following is a standard integral,
\[
\int_{-\infty}^{\infty} d x e^{-i x^{2}}=\sqrt{-\imath \pi}
\]

Therefore we get,
$$ \left\langle x^{\prime}\left|e^{-\frac{\imath}{\hbar} H t}\right| x\right\rangle= e^{\frac{i}{\hbar} \frac{\left(x-x^{\prime}\right)^{2}}{4 \alpha^{2}}}  \frac{1}{2 \pi \hbar} \frac{\sqrt{-\imath \pi}}{\alpha}  $$
Putting the value of $\alpha^{2} = \frac{t}{2 m}$ , we get the result stated above.



\end{document}