\documentclass{article}
\usepackage{amsmath}
\usepackage{amssymb}
\usepackage[utf8]{inputenc}
\usepackage{geometry}
\geometry{margin=1 in}


\title{Assignment 4}
\author{Mytraya Gattu, Roshni Singh, Ramakant Pal\\ 180050032, 18B030020, 180260028}
\date{\today}

\begin{document}

\maketitle
Under the infintesimal transformation, $x^{i}$ and $p^{i}$ change as:
$$\tilde{x}^{i}=x^{i}+\epsilon \frac{\partial g(\bar{x},\bar{p})}{\partial p^{i}}$$
$$\tilde{p}^{i}=p^{i}-\epsilon \frac{\partial g(\bar{x},\bar{p})}{\partial x^{i}}$$
Taking $g(\bar{x},\bar{p})$ to be $H(\bar{x},\bar{p})$, and from the equations of motion
$$\dot{p}^{i} = -\frac{\partial H(\bar{x},\bar{p})}{\partial x^{i}}$$
$$\dot{x}^{i} = \frac{\partial H(\bar{x},\bar{p})}{\partial p^{i}}$$
the transformed co-ordinates are:
$$\tilde{x}^{i}=x^{i}+\epsilon \dot{x}^{i} = x^{i}(t + \epsilon)$$
$$\tilde{p}^{i}=p^{i}-\epsilon \left(-\dot{p}^{i}\right) = p^{i}+\epsilon \dot{p}^{i} = p^{i}(t+\epsilon)$$
Hence,this transformation corresponds to a translation in time. 

\end{document}